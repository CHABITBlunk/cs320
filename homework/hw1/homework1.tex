\documentclass{article}
\title{Homework 1}
\author {Habit Blunk}
\date{February 2025}
\begin{document}
\maketitle
\begin{enumerate}
  \item Base case. Begin by proving the first few cases
    \begin{itemize}
      \item 18 cents
        \begin{displaymath}
          3 + 3 + 3 + 3 + 3 + 3 = 18
        \end{displaymath}
      \item 19 cents
        \begin{displaymath}
          10 + 3 + 3 + 3 = 19
        \end{displaymath}
      \item 20 cents
        \begin{displaymath}
          10 + 10 = 20
        \end{displaymath}
      \item 21 cents
        \begin{displaymath}
          3 + 3 + 3 + 3 + 3 + 3 + 3 = 21
        \end{displaymath}
      \item 22 cents
        \begin{displaymath}
          10 + 3 + 3 + 3 + 3 = 22
        \end{displaymath}
    \end{itemize}
    Thus, any fee from 18-22 cents can be paid for using only 3-cent and 10-cent stamps.
  \item Inductive hypothesis\\
    Assume that for some postage fee $ n >= 18 $, we can pay for this using only 3-cent and 10-cent stamps.
  \item Inductive step\\
    Prove that if we can pay for $ n + 1 $ cents, we can also pay for $ n - 2 $ cents.
    If $ n - 2  >= 18 $, then all we have to do to pay for a postage fee of $ n + 1 $ cents is add another 3-cent stamp.
    \begin{displaymath}
      n + 1 = (n - 2) + 3 cents
    \end{displaymath}
  \item Conclusion\\
    Therefore, we can pay for any postage fee over 18 cents using only 3-cent and 10-cent stamps.
\end{enumerate}
\end{document}
